\documentclass{article}
\usepackage[utf8]{inputenc}

\title{FOAR705 Elaboration I: Planning}
\author{Ellen Kirkpatrick }
\date{August 2019}

\begin{document}

\maketitle

\section{Research Area}
Great inequality exists in welfare outcomes across different population groups in Australia despite the implementation of the modern welfare system in the 1970s. To what extent does the welfare system embody the political tensions within the Australian context and has this led to the continued marginalisation and differentiation of these groups?

\section{Scoping problem}
A major problem in my research is being able to identify existing connections between sources from different databases based on key themes. The end goal is a program which can access multiple sources at once and highlight any common themes or terms across them. It would also be able to store metadata and annotations to make use of them in future more efficient

\section{Decomposition: finding and identifying connections between multiple qualitative sources}
\begin{itemize}
    \item Identify key themes and terms relevant to my research question. This may include welfare, Australia, politics, ideology.
    \item Enter these terms in Macquarie University library database and select multiple sources which appear to be relevant to these key terms. 
    \item Copy url details and commit these sources to  voyant-tools.org. 
    \item If Voyant is sufficient and is chosen as the program, use the trends and summary tools to identify recurring themes or patterns which are common across the different sources. 
    \item Use the reader tool on Voyant to read through abstracts to determine whether the source may be relevant for research project or a specific part of the research project.
    \item Once the relevance or topic area is determined, export the source and reference to zotero and test whether successful. 
    \item If Zotero works adequately, add annotations for each source.
    \item Create tags for each source on zotero according to key themes and to create connections between sources.
    \item Access sources from zotero for further reading, interpretation and qualitative analysis. 
\end{itemize}

\section{Elaboration: what needs to be tested }
\begin{itemize}
    \item Test whether Macquarie University library database provides the relevant metadata required for Harvard referencing system.
    \item If not successful, test another database. 
    \item Test whether Voyant is successful in identifying common themes or patterns within well known sources. 
    \item If Voyant successfully identifies themes or terms that are known to be in sources, test on sources which are new and unknown. \item If Voyant is chosen as a useful online platform, test whether Zotero is able to extract metadata from sources in Voyant.
    \item If Voyant is not chosen, test whether Zotero can extract metadata from the university database. 
    \item Test whether metadata and source is successfully imported into Zotero.
    \item Find where source and metadata is stored.
    \item Test adding annotations to specific sources in Zotero.
    \item Find how to access annotations.
    \item Test whether tags can be added to multiple sources linking them together in Zotero.
    \item Test whether multiple sources can be searched and accessed at once according to their tag. 
\end{itemize}


\end{document}
