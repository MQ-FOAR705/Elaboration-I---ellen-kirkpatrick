\documentclass{article}
\usepackage[utf8]{inputenc}

\title{FOAR705 Elaboration I: Planning}
\author{Ellen Kirkpatrick }
\date{August 2019}

\begin{document}

\maketitle

\section{Research Area}
Great inequality exists in welfare outcomes across different population groups in Australia despite the implementation of the modern welfare system in the 1970s. To what extent does the welfare system embody the political tensions within the Australian context and has this led to the continued marginalisation and differentiation of these groups?

\section{Scoping problem}
The major scoping problem identified in the week 4 scoping exercise were difficulties in reading multiple sources from different databases and identifying existing connections between them based on key themes. The decomposition and algorithm design demonstrated a step-by-step breakdown through which the end goal was a singular program which could access multiple sources at once. This program would draw connections between sources based on recurring themes and terms. It would also export referencing details to make storage of the reference list easier throughout the entire project. \\
For the purposes of elaboration, the scoping has been narrowed down to finding technologies or programs which can draw connections between sources based on key themes and then being able to store these sources accordingly with their reference details. \\
The programs chosen to overcome these problems and accomplish the scoping tasks are Voyant, Zotero and Cloudstor. 

\section{Revised scoping steps}
\begin{itemize}
    \item Identify key terms.
    \item Use key words to search databases.
    \item Find sources.
    \item Commit sources to Voyant.
    \item Identify recurring themes.
    \item Read abstracts.
    \item Decide relevance.
    \item Export sources to cloudstor and zotero for storage.
    
\end{itemize}

\section{Elaboration}
\begin{itemize}
    \item Identify key themes and terms which are thought to be relevant to my research question. These may include terms such as welfare system, Australia, privilege, ideology, politics, disadvantage, benefits, population groups, society.
    \item Enter these terms in Macquarie University library database and select multiple sources which appear to be relevant to these key terms. \item Copy url details and commit these sources to  voyant-tools.org. 
    \item On Voyant, use the trends and summary tools to identify recurring themes or patterns which are common across the different sources. 
    \item Use the reader tool on Voyant to read through abstracts to determine whether the source may be relevant for research project or to narrow down which part of the project they are relevant for.
    \item Once the relevance or topic area is determined, export the source and reference to zotero and add metadata for each source.
    \item Create tags for each source on zotero according to key themes and to create connections between sources.
    \item Import sources to cloudstor for additional non-localised storage.
    \item Access sources from zotero or cloudstor for further reading, interpretation and analyse. 
\end{itemize}





\end{document}
