\documentclass{article}
\usepackage[utf8]{inputenc}

\title{FOAR705 Elaboration I: Planning}
\author{Ellen Kirkpatrick }
\date{August 2019}

\begin{document}

\maketitle

\section{Research Area}
Great inequality exists in welfare outcomes across different population groups in Australia despite the implementation of the modern welfare system. To what extent does the welfare system embody the political tensions within the Australian context and has this led to the continued marginalisation and differentiation of these groups?

\section{Scoping problem}
The scoping problem identified in week 4 was difficulties in reading multiple sources from different databases and to identify existing connections based on key themes. The initial goal and process outlined through the decomposition and algorithm design was to find a singular program within which multiple sources could be accessed at once. In this program, it would be possible to draw links between sources based on key terms and themes. The initial scoping goal also included cross-checking reference lists and citations, however, I have narrowed the focus only to exporting the reference if the source is decided as relevant. These are more prominent issues in my research.
\section{Revised scoping steps}
\begin{itemize}
    \item Identify key terms.
    \item Use key words to search databases.
    \item Find sources.
    \item Commit sources to Voyant.
    \item Identify recurring themes.
    \item Read abstracts.
    \item Decide relevance.
    \item Export sources to cloudstor.
    \item Store in cloudstor according to topic area.
    \item Export references to endnote. 
    \end{itemize}

\section{Elaboration}
\begin{itemize}
    \item Identify key themes and terms which are thought to be relevant to my research question. These may include terms such as welfare system, Australia, privilege, ideology, politics, disadvantage, benefits, population groups, society.
    \item Use these terms to conduct primary database search.
    \item Drag and drop urls of initial sources found onto Voyant-tools.org.
\end{itemize}





\end{document}
